\documentclass[polish,aspectratio=169]{beamer}

% wide screen
% \documentclass[aspectratio=169]{beamer}


%%% YOUR PACKAGES HERE %%%
\usepackage{comment}
\usepackage{hyperref}
\usepackage[backend=biber, style=numeric, bibstyle=ieee,
sorting=none, isbn=false, urldate=ymd,
doi=false, url=true]{biblatex}
\addbibresource{bibliography/bibliography.bib}
    \renewbibmacro{in:}{}
% \usepackage{tikz} 
% \usetikzlibrary{graphs,graphs.standard,automata}


% polish language
\usepackage[polish]{babel}
\usepackage{polski}



%%% IMPORT PG PRESENTATION STYLE %%%
% THIS IS GDANSK UNIVERSITY OF TECHNLOGOGY (PG) PRESENTATION TEMPLATE
% Creator: Jan Cychnerski <jan.cychnerski@eti.pg.edu.pl>
% Copyleft 2019


% PG THEME OPTIONS

\usetheme{Boadilla}
\usecolortheme{default}
\usefonttheme{professionalfonts}

% colors

\definecolor{PGBlue}{RGB}{0,56,101}
\definecolor{PGRed}{RGB}{193,10,39}
\definecolor{PGSilver}{RGB}{200,200,200}
\definecolor{PGBlack}{RGB}{0,0,0}

% PGBlue
\setbeamercolor{frametitle}{fg=PGBlue}
\setbeamercolor{normal text}{fg=PGBlue}
\setbeamercolor{structure}{fg=PGBlue}
\setbeamercolor{item}{fg=PGBlue}

% PGRed
\setbeamercolor{alerted text}{fg=PGRed}
\setbeamercolor{item projected}{fg=PGRed}

% white
\setbeamercolor{title}{fg=white}
\setbeamercolor{titlelike}{fg=white}
\setbeamercolor{subtitle}{fg=white}

% enumerate and itemize styles

\setbeamertemplate{itemize item}{\bfseries\color{PGRed}\raise1pt\hbox{\donotcoloroutermaths$\bullet$}}
\setbeamertemplate{itemize subitem}{\color{PGRed}\raise0.5pt\hbox{--}}
\setbeamertemplate{itemize subsubitem}{\color{PGRed}\tiny\raise1.5pt\hbox{\donotcoloroutermaths$\bullet$}}

\setbeamertemplate{enumerate item}{\bfseries\color{PGRed}\insertenumlabel.}
\setbeamertemplate{enumerate subitem}{\color{PGRed}\insertsubenumlabel.}
\setbeamertemplate{enumerate subsubitem}{\color{PGRed}\insertsubsubenumlabel.}
\setbeamertemplate{enumerate mini template}{\insertenumlabel}


% disable navigation

\beamertemplatenavigationsymbolsempty

% additional commands

\newcommand*{\vcenteredhbox}[1]{\begingroup
\setbox0=\hbox{#1}\parbox{\wd0}{\box0}\endgroup}

\graphicspath{{pgbeamer/}}


\usepackage{iflang}
\IfLanguageName{polish}{
\newcommand{\pglogobig}{pg-logo-big-pl}
\newcommand{\pglogosmall}{pg-logo-small-pl}
}{
\newcommand{\pglogobig}{pg-logo-big-en}
\newcommand{\pglogosmall}{pg-logo-small-en}
}


% FRAME TITLE LOGO
\addtobeamertemplate{frametitle}{\vcenteredhbox{\includegraphics[height=8mm]{\pglogosmall}}\bfseries}{}


\newcommand{\pgtitleframe}{{
% PG TITLE PAGE

\setbeamercolor{background canvas}{bg=PGBlue}
\setbeamercolor{title}{fg=white}
\setbeamercolor*{date}{fg=white}
\setbeamercolor*{author}{fg=white}

\setbeamertemplate{footline}{}

\begin{frame}[noframenumbering]
\centering
\vspace{1cm}
\includegraphics[height=3cm]{\pglogobig}
\vspace{5mm}
\maketitle
\end{frame}
}}

\newcommand{\pglastframe}{{
% PG LAST PAGE

\setbeamercolor{background canvas}{bg=PGBlue}
\setbeamercolor{title}{fg=white}
\setbeamercolor*{date}{fg=white}
\setbeamercolor*{author}{fg=white}

\setbeamertemplate{footline}{}

\begin{frame}[noframenumbering]
\centering
\vspace{1cm}
\includegraphics[height=5cm]{\pglogobig}
\end{frame}
}}



%%% YOUR OPTIONS HERE %%%

\title[Property based testing]{Property based testing}
\subtitle{Jak testować zachowanie, a nie przypadki}
\author{Paulina Brzęcka \and Marek Borzyszkowski}
\date{\today}

\setbeamercovered{transparent}

\begin{document}
%%% PG TITLE PAGE %%%
\pgtitleframe

%%% YOUR PRESENTATION HERE %%%

\setbeamercovered{invisible}

\begin{frame}{Wstęp}
    Istnieje wiele koncepcji testowania oprogramowania, jedną z nich jest testowanie na podstawie właściwości \cite{pbt_bib}.
    \onslide<2->{Dziś dowiecie się:
        \begin{enumerate}
        \item na czym polega testowanie na podstawie właściwości.
        \item Czym różni się ono od klasycznego podejścia do testowania.
        \item Jakie są strategie testowania na podstawie właściwości. 
        \item Jak przy wykorzystaniu konceptu QuickCheck znaleźć wartości początkowe nieprzechodzące testy.
        \end{enumerate}
    }
\end{frame}

\begin{frame}{Przykład}
    TODO
\end{frame}

\begin{frame}{Strategie}
    Testowanie na podstawie właściwości nie jest proste w zastosowaniu, przynajmniej przy pierwszej próbie.
    Istnieją jednak pewne schematy wskazujące drogę, jak stworzyć takie testy.

\end{frame}


\begin{frame}{Pytania}
    \begin{center}
        {\huge Pytania?}
    \end{center}
\end{frame}

\setbeamercovered{transparent}
\begin{frame}[allowframebreaks]{Bibliografia}
    \printbibliography
\end{frame}

\begin{frame}{Podziękowania}
    Chcielibyśmy podziękować Panu dr. inż. Janowi Cychnerskiemu za stworzenie 
    i udostępnienie stylu \href{https://github.com/jachoo/pg-beamer}{\emph{pg-beamer}}, 
    co zostało wykorzystane do stworzenia tej prezentacji.\\
    \url{https://github.com/jachoo/pg-beamer}
     
\end{frame}

\begin{frame}{Koniec}
    \begin{center}
        {\huge Dziękujemy za uwagę!}
    \end{center}
\end{frame}

%%% PG LAST PAGE %%%
\pglastframe


%%% DOCUMENT ENDS HERE. Good bye! :) %%%

\end{document}

\documentclass[polish,aspectratio=169]{beamer}

% wide screen
% \documentclass[aspectratio=169]{beamer}


%%% YOUR PACKAGES HERE %%%
\usepackage{comment}
\usepackage{hyperref}
% \usepackage{tikz} 
% \usetikzlibrary{graphs,graphs.standard,automata}


% polish language
\usepackage[polish]{babel}
\usepackage{polski}



%%% IMPORT PG PRESENTATION STYLE %%%
\include{pgbeamer/pgbeamer}


%%% YOUR OPTIONS HERE %%%

\title[Property based testing]{Property based testing}
\subtitle{Jak testować zachowanie, a nie przypadki}
\author{Paulina Brzęcka \and Marek Borzyszkowski}
\date{\today}

\setbeamercovered{transparent}

\begin{document}
%%% PG TITLE PAGE %%%
\pgtitleframe

%%% YOUR PRESENTATION HERE %%%

\setbeamercovered{invisible}


% \begin{frame}{Policjanci i złodziej}
%     Gra odbywa się na grafie $G$ między drużyną policjantów i złodziejem. 
%     Drużyna policjantów składa się z $K>0$ policjantów, gdzie $K$ to liczba policjantów.
%     Policjanci i złodzieje znajdują się na wierzchołkach grafu $V\left(G\right)$. \pause
%     W grze można wyróżnić 2 fazy:
%     \begin{enumerate}
%         \item fazę umieszczania najpierw wszystkich policjantów, a potem złodzieja na wierzchołkach $V\left(G\right)$.
%         \item Fazę ruchu, która odbywa się turowo, najpierw ruszają się wszyscy policjanci, potem turę ma złodziej. 
%         Dozwolonymi ruchami są:
%         \begin{enumerate}
%             \item brak zmiany wierzołka,
%             \item zmiana wierzchołka na inny będący połączony krawędzią z wierzchołkiem, 
%             na którym aktualnie przebywa dana osoba.
%         \end{enumerate} 
%     \end{enumerate}
% \end{frame}


\begin{frame}{Pytania}
    \begin{center}
        {\huge Pytania?}
    \end{center}
\end{frame}

\setbeamercovered{transparent}
\begin{frame}[allowframebreaks]{Bibliografia}
    \bibliographystyle{plain}
    \bibliography{bibliography/bibliography.bib}
\end{frame}

\begin{frame}{Podziękowania}
    Chcielibyśmy podziękować Panu dr. inż. Janowi Cychnerskiemu za stworzenie 
    i udostępnienie stylu \href{https://github.com/jachoo/pg-beamer}{\emph{pg-beamer}}, 
    co zostało wykorzystane do stworzenia tej prezentacji.\\
    \url{https://github.com/jachoo/pg-beamer}
     
\end{frame}

\begin{frame}{Koniec}
    \begin{center}
        {\huge Dziękujemy za uwagę!}
    \end{center}
\end{frame}

%%% PG LAST PAGE %%%
\pglastframe


%%% DOCUMENT ENDS HERE. Good bye! :) %%%

\end{document}

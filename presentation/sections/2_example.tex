\begin{frame}{Testowanie na przykładach}
    \begin{itemize}
        \item Testowanie funkcji na podstawie konkretnych wartości wejściowych i oczekiwanych rezultatów.
    \end{itemize}
    \end{frame}
    
\begin{frame}[fragile]{Kod testujący}
    \begin{lstlisting}[language=FSharp, xleftmargin=-10pt,xrightmargin=-10pt,numbers=none]
    [Test]
    let ``Add two numbers, expect their sum``() =
        let testData = [(1,2,3); (2,2,4); (3,5,8); (27,15,42)]
        for (x,y,expected) in testData do
            let actual = add x y
            Assert.AreEqual(expected,actual)
    \end{lstlisting}
\end{frame}
    
\begin{frame}[fragile]{Deweloper potrafi napisać coś takiego...}
    \begin{lstlisting}[language=FSharp, xleftmargin=-10pt,xrightmargin=-10pt,numbers=none]
        let add x y =
            match x,y with
            | 1,2 -> 3
            | 2,2 -> 4
            | 3,5 -> 8
            | 27,15 -> 42
            | _ -> 0
    \end{lstlisting}
    Dodatkowo odgrażając się, że podąża za regułą TDD, czyli poprawia swój kod, dopóki test nie będzie zielony :)
\end{frame}
    
\begin{frame}{Ograniczenia podejścia example-based}
    \begin{itemize}
        \item Musimy ręcznie wymyślić testowe dane wejściowe.
        \item Testy obejmują tylko przypadki, które sami zdefiniujemy.
        \item Trudno przewidzieć wszystkie możliwe scenariusze (np. wartości graniczne, liczby ujemne, bardzo duże liczby).
    \end{itemize}
\end{frame}

\begin{frame}[fragile]{Automatyczne generowanie danych}
    \begin{lstlisting}[language=FSharp, xleftmargin=-20pt,xrightmargin=-10pt,numbers=none]
    [Test]
    let ``Add two random numbers 100 times, expect their sum``() =
        for _ in [1..100] do
            let x = randInt()
            let y = randInt()
            let expected = x + y
            let actual = add x y
            Assert.AreEqual(expected,actual)
    \end{lstlisting}
\end{frame}
    

\begin{frame}{Testowanie na podstawie właściwości}
    \begin{itemize}
        \item Zamiast testować pojedyncze przypadki, definiujemy ogólne właściwości, które funkcja powinna spełniać.
        \item Narzędzia, np. \texttt{FsCheck}, generują dane wejściowe i sprawdzają właściwości.
    \end{itemize}
    \textbf{Przykładowe właściwości:}
    \begin{itemize}
        \item Przemienność.
        \item Dodanie liczby 1 dwukrotnie jest równe dodaniu liczby 2 raz.
        \item Dodanie liczby 0 nie zmienia wartości liczby.
    \end{itemize}
\end{frame}


\begin{frame}[fragile]{Przemienność}
    \begin{lstlisting}[language=FSharp, xleftmargin=-10pt,xrightmargin=-10pt,numbers=none]
    let add x y = x + y

    let commutativeProperty x y =
        let result1 = add x y
        let result2 = add y x 
        result1 = result2

    [Test]
    let addDoesNotDependOnParameterOrder() =
        Check.Quick commutativeProperty
    \end{lstlisting}
\end{frame}

\begin{frame}[fragile]{Pozostałe właściwości}
    \begin{lstlisting}[language=FSharp, xleftmargin=-10pt,xrightmargin=-10pt,numbers=none, basicstyle=\ttfamily\small]
    let add1TwiceIsAdd2Property x _ =
        let result1 = x |> add 1 |> add 1
        let result2 = x |> add 2
        result1 = result2

    [Test]
    let addOneTwiceIsSameAsAddTwo() =
        Check.Quick add1TwiceIsAdd2Property

    let identityProperty x _ =
        let result1 = x |> add 0
        result1 = x

    [Test]
    let addZeroIsSameAsDoingNothing() =
        Check.Quick identityProperty
    \end{lstlisting}
\end{frame}

\begin{frame}{Zalety podejścia property-based}
    \begin{itemize}
        \item Większa pewność poprawności implementacji.
        \item Lepsze zrozumienie wymagań i istoty problemu.
        \item Automatyczne generowanie danych pozwala na testowanie różnych scenariuszy.
    \end{itemize}
\end{frame}
\section{Wstęp}

bum tralala chlapie fala\cite{boinski2007kaskbook}

\begin{figure}
    \centering
    \includegraphics[width=0.7\textwidth]{images/obrazek.png}
    \caption{Opis do obrazka.}
    \label{fig:obrazek}
\end{figure}

\lstset{basicstyle=\scriptsize}
\begin{lstlisting}[frame=single,caption={Przykładowy listing},label=kod:listingA]
<?xml version="1.0"?>

<!DOCTYPE rdf:RDF [
 <!ENTITY owl "http://www.w3.org/2002/07/owl#" >
 <!ENTITY xsd "http://www.w3.org/2001/XMLSchema#" >
 <!ENTITY rdfs "http://www.w3.org/2000/01/rdf-schema#" >
 <!ENTITY rdf "http://www.w3.org/1999/02/22-rdf-syntax-ns#" >
]>

<rdf:RDF xmlns="http://kask.eti.pg.gda.pl/securityA.owl#"
 xml:base="http://kask.eti.pg.gda.pl/securityA.owl"
 xmlns:rdfs="http://www.w3.org/2000/01/rdf-schema#"
 xmlns:owl="http://www.w3.org/2002/07/owl#"
 xmlns:xsd="http://www.w3.org/2001/XMLSchema#"
 xmlns:rdf="http://www.w3.org/1999/02/22-rdf-syntax-ns#">
 <owl:Ontology rdf:about="http://kask.eti.pg.gda.pl/securityA.owl"/>

\end{lstlisting}

\begin{table}
    \caption{Opis tabelki}
    \label{tab:tab1}
     \centering
    \begin{tabular}{|c|p{4.5cm}|p{6cm}|}
      \hline
      Pole 1 & Pole 2 & Pole 3\\
      \hline
      1 & Pojęcie & Opis tegoż pojęcia\\
      \hline
      2 & Pojęcie & Opis tegoż pojęcia\\
      \hline
    \end{tabular}
    % \vspace{-0.5cm}
\end{table}
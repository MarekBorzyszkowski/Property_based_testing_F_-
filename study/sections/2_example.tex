\section{Geneza}

\section*{Testowanie na przykładach (Example-Based Testing)}

W testowaniu na przykładach sprawdzamy funkcję na podstawie konkretnych wartości wejściowych i oczekiwanych rezultatów.

\subsection*{Załóżmy, że chcemy przetestować funkcję dodawania.}

\subsubsection*{Testowanie na przykładach:}
\lstset{language=FSharp, basicstyle=\scriptsize}
\begin{lstlisting}[frame=single,caption={Testowanie na przykładach},label=kod:listingA]
[Test]
let ``Add two numbers, expect their sum``() =
    let testData = [ (1,2,3); (2,2,4); (3,5,8); (27,15,42) ]
    for (x,y,expected) in testData do
        let actual = add x y
        Assert.AreEqual(expected,actual)
\end{lstlisting}

Deweloper potrafi napisać coś takiego:
\lstset{language=FSharp, basicstyle=\scriptsize}
\begin{lstlisting}[frame=single,caption={Testowany kod},label=kod:listingA]
let add x y =
    match x,y with
    | 1,2 -> 3
    | 2,2 -> 4
    | 3,5 -> 8
    | 27,15 -> 42
    | _ -> 0
\end{lstlisting}

Dodatkowo odgrażając się, że podąża za regułą TDD, czyli poprawia swój kod minimalnie, dopóki test nie będzie zielony :)

\subsubsection*{Ograniczenia:}
\begin{itemize}
    \item Musimy ręcznie wymyślić testowe dane wejściowe.
    \item Testy obejmują tylko przypadki, które sami zdefiniujemy.
    \item Trudno przewidzieć wszystkie możliwe scenariusze (np. wartości graniczne, liczby ujemne, liczby bardzo duże itp.).
\end{itemize}

Można wymusić prawidłowy kod na podstawie generatora liczb losowych:
\lstset{language=FSharp, basicstyle=\scriptsize}
\begin{lstlisting}[frame=single,caption={Przykładowe rozwiązanie},label=kod:listingA]
[Test]
let ``Add two random numbers 100 times, expect their sum``() =
    for _ in [1..100] do
        let x = randInt()
        let y = randInt()
        let expected = x + y
        let actual = add x y
        Assert.AreEqual(expected,actual)
\end{lstlisting}

Natomiast w tym przypadku musimy zaimplementować w teście \texttt{add}, aby przetestować funkcję \texttt{add}.

\section*{Testowanie na podstawie właściwości (Property-Based Testing)}

Zamiast testować pojedyncze przypadki, definiujemy ogólne właściwości, które funkcja powinna zawsze spełniać. Są różne narzędzia, np. \texttt{FsCheck}, które automatycznie generują dane wejściowe i sprawdzają, czy właściwości są spełniane.

\subsection*{Przykład właściwości dla funkcji dodawania:}
\begin{itemize}
    \item Przemienność.
    \item Dodanie liczby 1 dwukrotnie jest tym samym co dodanie liczby 2 raz.
    \item Dodanie liczby 0 nie zmienia wartości liczby.
\end{itemize}

\lstset{language=FSharp, basicstyle=\scriptsize}
\begin{lstlisting}[frame=single,caption={Testowanie na właściwościach},label=kod:add_all_properties]
let add x y = x + y

let commutativeProperty x y =
    let result1 = add x y
    let result2 = add y x 
    result1 = result2

[Test]
let addDoesNotDependOnParameterOrder() =
    Check.Quick commutativeProperty

let add1TwiceIsAdd2Property x _ =
    let result1 = x |> add 1 |> add 1
    let result2 = x |> add 2
    result1 = result2

[Test]
let addOneTwiceIsSameAsAddTwo() =
    Check.Quick add1TwiceIsAdd2Property

let identityProperty x _ =
    let result1 = x |> add 0
    result1 = x

[Test]
let addZeroIsSameAsDoingNothing() =
    Check.Quick identityProperty
\end{lstlisting}

Dzięki podejściu \textit{property-based} mamy większą pewność, że implementacja jest właściwa. Dodatkowo to podejście pozwala lepiej zrozumieć wymagania i istotę całego problemu.\\
Zdefiniowanie właściwości w bardziej złożonych przypadkach potrafi być bardzo problematyczne. Natomiast istnieją pewne techniki, które pomagają w ich definiowaniu.